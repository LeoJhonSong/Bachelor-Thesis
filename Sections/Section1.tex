\section{Introduction}

\subsection{Motivation}

People's demands on seafood with high nutrition such as sea urchin, sea cucumber
and scallop, are increasing rapidly these days. However, because of the
requirements of breeding environment, these seafoods are currently kept in
offshore huge cages at relatively deep sea floor (more than 10 meters depths)
and harvested artificially by seafood divers, which are also called `Haipengzi'
in some parts of China \parencite{hpz}. The seafood divers not only suffer from
long working time (about 12hours) but also risk psychological stress, life
hazards, hypothermia disease and decompression illness
\parencite{barratt2002decompression}. Besides, the maintenance of boat, diving
equipments costs a lot \parencite{divervd}.

Therefore, robot solutions for such submarine operations are urgently needed to
decrease diver related injury, reduce harvest cost and improve manipulation
efficiency. Several \gls{rov} solutions have already been carried out recent
years. For instance, a modified \gls{rov} from Nomia, whose capture is
implemented by vacuum absorption controlled by operator onboard the vessel,
achieved a notable amount of 6595kg export quality urchins out of total harvest
of 1.88 ton in 4.5 days \parencite{james2012test}. It shows out the advantage of
robot in place of divers when it comes to the cold working conditions and dark
period working. The disadvantage of this solution is that the vacuum absorption
may result in submarine vegetation environment damage. In addition, its bycatch
is still too much, result in relatively low efficiency.

There are a big amount of \gls{rov}s designed for submarine rescue, observation
and similar tasks, including oceanic bottom flying node
\parencite{qin2019distributed}, H300 ROV of French ECA group (ECA group). These
\gls{rov}s share some common features:
\begin{itemize}
    \item the operations are realized by pilot through remote control
    \item launched from specific mother vessel with an umbilical cable for
    communication and energy supply
    \item large and heavy for transportation and handling
    \item has complex interface which may lead to pilot fatigue after long hours
    of operations \parencite{6003619}
\end{itemize}

Different from \gls{rov}s, \gls{AUV}s, such as SAUVIM AUV
\parencite{marani2014introduction}, Girona 500 I-AUV \parencite{ribas2015auv}
could achieve environment perception, analysis and operative tasks autonomously
and independently in complicated environment \parencite{barbualatua2014dynamic}.
Therefore, National Natural Science Foundation of  China (NSFC) launched a group
of key projects of underwater vehicle environmental perception and target
capture to realize robotic capturing in 2016. The main objectives of these
projects are to develop low cost underwater robot capturing platforms on the
basis of the following researches:
\begin{enumerate}
    \item environmental perception of  underwater landscapes, plants and
    cur-rents
    \item rapid detection, recognition and tracking of target organism
    \item aliveness and rapid capture of target organism
\end{enumerate}

From these researches, not only the oceanic organism harvest is expected to be
mechanized, but also the underwater rescue, oceanic engineering and submarine
exploring are propelled.

This article tries to give solution to following questions:

\begin{enumerate}
    \item How to integrate soft manipulator into mobile remote controlled
    autonomous underwater vehicle robot
    \item How to find underwater targets and then collect them safely
    \item How to achieve seafood collecting tasks in complex water environment
\end{enumerate}

% \subsection{Challenges}

% However, in order to realize autonomous underwater operation, researchers should
% not only improve robot intelligence, precise control and environmental
% perception, but also overcome the disturbances from manipulator and oceanic
% current \parencite{zhang2012vision}.

% The  realization of  underwater robotic organism capture missions requires
% overcoming some novel difficulties. Since the lights are scat-tered and absorbed
% by silt, hydrone and particles, the images are grayed, obscured contrast reduced
% (Gianluca Antonelli et  al.,  2014).  On  the other hand, the sea cucumbers are
% soft and smooth organism, sea shell absorbs itself on the rock for the
% protection once scared, both of them living close to the rocks with appearances
% resemble to their circum-stance. It is difficult for these existing underwater
% robots to autono-mously percept complicated organism habitat and afterward,
% realize rapid and uninjured grasp.

% \subsection{Thesis Outline}
